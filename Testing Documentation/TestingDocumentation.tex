\documentclass[a4paper,12pt]{report}
\addtolength{\oddsidemargin}{-1.cm}
\addtolength{\textwidth}{2cm}
\addtolength{\topmargin}{-2cm}
\addtolength{\textheight}{3.5cm}
\newcommand{\HRule}{\rule{\linewidth}{0.5mm}}
\setcounter{secnumdepth}{5}
\setcounter{tocdepth}{3}
\makeindex

\usepackage{longtable}
\usepackage{graphicx}
\usepackage{makeidx}
\usepackage{hyperref}
\usepackage{verbatim}

\hypersetup{
    colorlinks=true,
    linkcolor=blue,
    filecolor=magenta,      
    urlcolor=cyan,
}


% define the title
\author{Ambitious Design}
\title{Testing Documentation}
\begin{document}
\setlength{\parskip}{6pt}

% generates the title
\begin{titlepage}

\begin{center}
% Upper part of the page           
\textsc{\LARGE Willburg Outdoor PTY(ltd.)}\\[1.5cm]
\textsc{\Large Smart Image Identifier }\\[1.0cm]
\textsc{\Large Version 1.0 }\\[0.5cm]
% Title
\HRule \\[0.4cm]
{ \huge \bfseries Testing Documentation}\\[0.4cm]
\HRule \\[0.4cm]
% Author and supervisor
\begin{minipage}{0.4\textwidth}
\begin{flushleft} \large
\emph{Author:}\\
Stephen {Swanepoel}
\end{flushleft}
\end{minipage}
\begin{minipage}{0.4\textwidth}
\begin{flushright} \large
\emph{} \\
u11032091
\end{flushright}
\end{minipage}
\begin{minipage}{0.4\textwidth}
\begin{flushleft} \large
Dian {Veldsman}
\end{flushleft}
\end{minipage}
\begin{minipage}{0.4\textwidth}
\begin{flushright} \large
\emph{} \\
u12081095
\end{flushright}
\end{minipage}


{\large \today}
\end{center}
\end{titlepage}
\footnotesize
\normalsize

\renewcommand{\thesection}{\arabic{section}}
\newpage

\section {Introduction}
	\subsection {Purpose}
		The purpose of the document is to give an overview of the test performed on the Smart Image Identifier project and the overall test coverage provided by these tests.
	\subsection {Scope}
		Smart Image Identifier aims to provide the client with a system that is able to identify humans from provided images. The system only has software components.
		Every use case that is provided by the system will be tested.
		
\section {Requirements for Test}
	The listing below identifies those items (use cases, functional requirements, non-functional requirements) that have been identified as targets for testing. 
	This list represents what will be tested.

\section {Requirements for Test}
	\begin {itemize}
		\item Programming Languages: Java
		\item Testing Frameworks: JUnit
		\item Coding Environment: Eclipse
		\item Operating System: Any
	\end {itemize}

\section {Assumptions and Dependencies}	
	\begin {enumerate}
		\item Assumptions:
			\begin {itemize}
				\item A server that handles images exists.
				\item All dependensies work and are up to date.
				\item The client has infrastructure to run the system on.
			\end {itemize}

		\item Dependencies:
		\begin {itemize}
			\item JUnit
			\item mongo-java-driver
			\item OpenCV
		\end {itemize}
	\end {enumerate}


	\subsection {Function Testing}
	\subsection {Performance Testing}
		\begin {itemize}
			\item Verify response to process an image
			\item Verify response to identify a human in an image
		\end {itemize}
	\subsection {Load Testing}
		\begin {itemize}
			\item Verify system response with single request to Smart Image Identifier
			\item Verify system response with multiple simultaneous request to Smart Image Identifier
		\end {itemize}
	\subsection {Stress Testing}
		\begin {itemize}
			\item Verify system response under high levels requests
		\end {itemize}
%	\subsection {Security}
	
\section {Test Strategy}
	The Test Strategy presents the recommended approach to the testing of the software applications. The previous section on Test Requirements described what will be tested; this describes how it will be tested.
	The main considerations for the test strategy are the techniques to be used and the criterion for knowing when the testing is completed.
	
	\section {Testing Types}
		\subsection {Function Testing}
			Testing of the application should focus on any target requirements that can be traced directly to use cases. 
			The goals of these tests are to verify proper data acceptance and processing.
			\subsubsection {Test Objective:}
				Ensure proper processing.
			\subsubsection {Technique:}
				\begin {itemize}
					\item Execute each use case, service contract, or function, using both valid and invalid data, to verify the following
						\subitem The expected results occur when valid data is used.
						\subitem The appropriate error / warning messages are displayed when invalid data is used.
				\end {itemize}
			\subsubsection {Completion Criteria:}
				\begin {itemize}
					\item All planned tests have been executed.
				\end {itemize}
				
		\subsection {Performance Testing}
			Performance testing measures response times, transaction rates, and other time sensitive requirements. 
			The goal of Performance testing is to verify and validate the performance requirements have been achieved. 
			\subsubsection {Test Objective:}
					Validate System Response time for process image request under a the following two conditions:
					\begin {itemize}
						\item normal anticipated volume
						\item anticipated worse case volume
					\end {itemize}
			\subsubsection {Technique:}
				\begin {itemize}
					\item Use Test scripts or unit tests 
				\end {itemize}
			\subsubsection {Completion Criteria:}
				\begin {itemize}
					\item Single request: Successful completion of the test scripts without any failures and within a respectable time period
					\item Multiple request: Successful completion of the test scripts without any failures and within a respectable time period
				\end {itemize}

\iffalse				
		\subsection {Load Testing}
			\subsubsection {Test Objective:}
			\subsubsection {Technique:}
			\subsubsection {Completion Criteria:}
			
		\subsection {Stress Testing}
			\subsubsection {Test Objective:}
			\subsubsection {Technique:}
			\subsubsection {Completion Criteria:}
			
		\subsection {Security Testing}
			\subsubsection {Test Objective:}
			\subsubsection {Technique:}
			\subsubsection {Completion Criteria:}
\fi

\section {Unit Test}
	\subsection {Process Image}
		\subsubsection {PeopleDetect\_Test}
			\paragraph {GreyScale}
				\subparagraph {Test Case ID:}
					testGreyScale
				\subparagraph {Test Tile:}
					Verify if image is grey scaled
				\subparagraph {Description:}
					Test if image is grey scaled after is sent has been sent through GreyScale()
				\subparagraph {Pre-conditions:}
					Non-null coloured buffered image is sent through as parameter
				\subparagraph {Dependencies:}
					testGenerateMat				
				\subparagraph {Post-conditions:}
					Grey scaled image is stored in output folder
					
			\paragraph {EnlargeImage}
				\subparagraph {Test Case ID:}
					testEnlargeImage
				\subparagraph {Test Tile:}
					Verify image height and width
				\subparagraph {Description:}
					Test if image is enlarged after it is sent through EnlargeImage()
				\subparagraph {Pre-conditions:}
					Buffered image of any size
				\subparagraph {Dependencies:}
					testGenerateMat				
				\subparagraph {Post-conditions:}
					Image is enlarged to 640x480

			\paragraph {GenerateMat}
				\subparagraph {Test Case ID:}
					testGenerateMat
				\subparagraph {Test Tile:}
					Verify if Mat object is created
				\subparagraph {Description:}
					Test Mat object is creared and returned after it is sent through generateMat()
				\subparagraph {Pre-conditions:}
					Non-null buffered image
				\subparagraph {Dependencies:}
					None			
				\subparagraph {Post-conditions:}
					Mat object is created and is not null
					
			\paragraph {GenerateImage}
				\subparagraph {Test Case ID:}
					testGenerateImage
				\subparagraph {Test Tile:}
					Verify image file is created and the Mat object is stored within it
				\subparagraph {Description:}
					Test if file is created with Mat object and is stored in output folder
				\subparagraph {Pre-conditions:}
					Non-null buffered image 
				\subparagraph {Dependencies:}
					testGenerateMat				
				\subparagraph {Post-conditions:}
					Image file is stored in output folder and has a valid image
					
			\paragraph {ProcessImage}
				\subparagraph {Test Case ID:}
					testProcessImage
				\subparagraph {Test Tile:}
					Verify image file is created, processed and stored and if human is detected
				\subparagraph {Description:}
					Test file is created, processed and stored and if human is detected
				\subparagraph {Pre-conditions:}
					Non-null buffered image 
				\subparagraph {Dependencies:}
					\begin {itemize}
						\item testGenerateMat	
						\item testEnlargeImage
						\item testGenerateImage
					\end {itemize}
				\subparagraph {Post-conditions:}
					Image file is stored in output folder and human is detected
				
\end{document}	